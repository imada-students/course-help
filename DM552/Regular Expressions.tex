\hypertarget{regular-expressions}{%
\section{Regular Expressions}\label{regular-expressions}}

Grep{[}Mode{]} {[}Regular Expression{]} {[}Filename{]}

\hypertarget{modes}{%
\subsection{Modes:}\label{modes}}

\begin{itemize}
\item
  -h (Do not display filenames)
\item
  -i (Ignore case)
\item
  -l (List only filenames containing matching lines
\item
  -n (Precede each matching line with its line number
\item
  -v (Negate matches)
\item
  -x Match whole line only (fgrep only)
\item
  -e (expression - specify expression as option)
\item
  -f filename
\end{itemize}

\hypertarget{regular-expressions-1}{%
\subsection{Regular Expressions:}\label{regular-expressions-1}}

\begin{longtable}[]{@{}lll@{}}
\toprule
Regular Expression & Meaning & Example\tabularnewline
\midrule
\endhead
. & The dot expression matches any character. & ``o.o'' matches any line
with a character in between two o's\tabularnewline
{[}{]} & Character classes can be used to match any specific set of
characters. & b{[}eor{]}at matches beat, brat, boat\tabularnewline
{[}\^{}{]} & Character classes can be negated. & b{[}\^{}eor{]}at
matches all lines not containing beat, brat, boat\tabularnewline
\^{} & Is an anchor, and means beginning of the line & \^{}b{[}eor{]}at
would only match if beat,breat, or boat was in the start of the
line.\tabularnewline
\$ & Is an anchor, means end of the line. & b{[}eor{]}at\$ would only
match if the 3 words where in the end of the line.\tabularnewline
\^{}\$ & Empty lines &\tabularnewline
\^{}word\$ & Only matches if the word is alone on a line
&\tabularnewline
* & The * is used to define zero or more occurrences. & Ya*y matches
yaaaaaaay, and yy.\tabularnewline
\begin{minipage}[t]{0.30\columnwidth}\raggedright
+\strut
\end{minipage} & \begin{minipage}[t]{0.30\columnwidth}\raggedright
The plus (+) means ``one or more''.

Equivalent to \{1,\}\strut
\end{minipage} & \begin{minipage}[t]{0.30\columnwidth}\raggedright
abc+d will match `abcd' , `abccd' , or `abccccccd' but will not match
`abd'.\strut
\end{minipage}\tabularnewline
\begin{minipage}[t]{0.30\columnwidth}\raggedright
?\strut
\end{minipage} & \begin{minipage}[t]{0.30\columnwidth}\raggedright
The `?' (question mark) specifies an optional character, the single
character that immediately precedes it

Equivalent to \{0,1\}\strut
\end{minipage} & \begin{minipage}[t]{0.30\columnwidth}\raggedright
July? will match `Jul' or `July'\strut
\end{minipage}\tabularnewline
\begin{minipage}[t]{0.30\columnwidth}\raggedright
\textbackslash{}n\strut
\end{minipage} & \begin{minipage}[t]{0.30\columnwidth}\raggedright
Backreference specifier, where n is a number. Looks for nth
subexpression\strut
\end{minipage} & \begin{minipage}[t]{0.30\columnwidth}\raggedright
For example, to find if the first word of a line is the same as the
last:

\^{}\textbackslash{}({[}{[}:alpha:{]}{]}\textbackslash{}\{1,\textbackslash{}\}\textbackslash{})
.* \textbackslash{}1\$\strut
\end{minipage}\tabularnewline
\bottomrule
\end{longtable}

*, ?, and + are known as quantifiers

Quantifiers can be used with subexpressions.

\begin{itemize}
\item
  (a*c)+ will match `c' , `ac' , `aac' or `aacaacac' but will not match
  `a' or a blank line
\end{itemize}

\hypertarget{more-about-character-classes}{%
\section{More about character
classes:}\label{more-about-character-classes}}

\includegraphics[width=4.10417in,height=2.55088in]{media/image1.png}

Named character classes are: alpha, lower, upper, alnum, digit, punct,
cntrl.

\begin{longtable}[]{@{}ll@{}}
\toprule
{[}{[}:alpha:{]}{]} - Named character classes &
{[}a-zA-Z{]}\tabularnewline
\midrule
\endhead
{[}{[}:alnum:{]}{]} & {[}a-zA-Z0-9\tabularnewline
{[}45{[}:lower:{]}{]} & {[}45a-z{]}\tabularnewline
\bottomrule
\end{longtable}

\hypertarget{repetition-ranges}{%
\section{Repetition ranges}\label{repetition-ranges}}

\begin{longtable}[]{@{}ll@{}}
\toprule
\{\} & notation can specify a range of repetitions for the immediately
preceding regex\tabularnewline
\midrule
\endhead
\{n\} & Means exactly n occurrences\tabularnewline
\{n,\} & Means at least n occurrences\tabularnewline
\{n,m\} & means at least n occurrences but no more than m
occurrences\tabularnewline
\bottomrule
\end{longtable}

.\{0,\} same as .*

a\{2,\} same as aaa*

\hypertarget{subexpressions}{%
\section{Subexpressions}\label{subexpressions}}

If you want to group part of an expression so that * or \{ \} applies to
more than just the previous character, use ( ) notation

Subexpresssions are treated like a single character

a* matches 0 or more occurrences of a

abc* matches ab, abc, abcc, abccc, \ldots{}

(abc)* matches abc, abcabc, abcabcabc, \ldots{}

(abc)\{2,3\} matches abcabc or abcabcabc
